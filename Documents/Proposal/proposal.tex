\documentclass[notitlepage]{report}

\usepackage[left=1in, right=1in, top=1in, bottom=1in]{geometry}
\usepackage[utf8]{inputenc}
\usepackage[english]{babel}

\usepackage{nth}
\usepackage{titling}


\pretitle{\begin{center}\Huge\bfseries}
\posttitle{\par\end{center}\vskip 0.5em}
\preauthor{\begin{center}\Large\ttfamily}
\postauthor{\end{center}}
\predate{\par\large\centering}
\postdate{\par}

\title{Artificial Intelligence and Feature Matching}
\author{Dustin Cook, Garfield Tong}
\date{March \nth{6} 2019}
\begin{document}

\maketitle
\thispagestyle{empty}

\begin{abstract}
This project will demonstrate the capability to teach neural networks how to find and match features between two images. It will explore the use of TensorFlow and feature matching within consumer applications. 
\end{abstract}

\begin{thebibliography}{9}

http://www.deeplearning.net/tutorial/dA.html

  https://www.groundai.com/project/unsupervised-sentence-compression-using-denoising-auto-encoders/1


https://codeburst.io/deep-learning-types-and-autoencoders-a40ee6754663

  [1] Moller, M. F. “A Scaled Conjugate Gradient Algorithm for Fast Supervised Learning”, Neural Networks, Vol. 6, 1993, pp. 525–533.


[2] Olshausen, B. A. and D. J. Field. “Sparse Coding with an Overcomplete Basis Set: A Strategy Employed by V1.” Vision Research, Vol.37, 1997, pp.3311–3325.


\end{thebibliography}


\end{document}
