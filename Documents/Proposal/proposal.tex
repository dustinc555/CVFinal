\documentclass[notitlepage]{report}

\usepackage[left=1in, right=1in, top=1in, bottom=1in]{geometry}
\usepackage[utf8]{inputenc}
\usepackage[english]{babel}

\usepackage{nth}
\usepackage{titling}


\pretitle{\begin{center}\Huge\bfseries}
\posttitle{\par\end{center}\vskip 0.5em}
\preauthor{\begin{center}\Large\ttfamily}
\postauthor{\end{center}}
\predate{\par\large\centering}
\postdate{\par}

\title{Artificial Intelligence and Feature Matching}
\author{Dustin Cook, Garfield Tong}
\date{March \nth{6} 2019}
\begin{document}

\maketitle
\thispagestyle{empty}

\begin{abstract}
This project will demonstrate the capability to teach neural networks how to find and match features between two images. It will explore the use of TensorFlow and feature matching within consumer applications. 
\end{abstract}

\section*{Proposal}
The purpose of this project is to provide a proof of concept. The aim is to show that a neural network can be trained to detect and match features between images. Feature matching is a topic explored within Computer Vision and has many applications, including recognition of protein strands on the molecular level, or object categorization by robots that are blasted off into space for sample collection. More over, on the consumer side, it's can be used for face detection, and panoramas. Artificial Intelligence has also become quite large. It is large enough for it to be the buzz word, and many tech news articles are excited to exclaim things such as, "The Honor 8X with AI Camera First Impressions" and "Essential Phone 2 may still happen, Essential reportedly making an AI phone." This project will use TensorFlow, a free and open-source application for training neural networks, to teach AI how to find and match features between two images. The inspiration for this idea comes from one of the newer features of smartphones, the dual back cameras. Generally, these phones have a color camera with lower pixel density, and a black and white camera with higher pixel density, and it combines them for a nice HD photograph. AI could do the feature detecting and matching and reduce the labor of combining the two images.

\section*{Steps}
\begin{enumerate}
  \item Generate test set and learning set. 
  \item Train the TensorFlow neural network with these sets of data.
  \item Test accuracy of the resulting algorithm.
  \item Make conclusion.
\end{enumerate}

\begin{thebibliography}{9}

\bibitem{website}
Goldsborough, P. and Informatik, F. (2016). A Tour of TensorFlow. [online] Arxiv.org. Available at: https://arxiv.org/pdf/1610.01178.pdf [Accessed 7 Mar. 2019].

\bibitem{website}
Usenix.org. (2016). TensorFlow: A System for Large-Scale Machine Learning. [online] Available at: https://www.usenix.org/system/files/conference/osdi16/osdi16-abadi.pdf [Accessed 7 Mar. 2019].

\bibitem{website}

Girshick, R., Donahue, J., Darrell, T. and Malik, J. (2014). Rich feature hierarchies for accurate object detection and semantic segmentation. [online] Openaccess.thecvf.com. Available at: http://openaccess.thecvf.com/content_cvpr_2014/papers/Girshick_Rich_Feature_Hierarchies_2014_CVPR_paper.pdf [Accessed 7 Mar. 2019].

\end{thebibliography}


\end{document}